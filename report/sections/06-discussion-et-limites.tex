\section{Discussions et limites}
\label{sec:discussion}

\paragraph{Hypothèses.} Entiers non négatifs ; formats 32 bits ; accès direct prioritaire ; en-tête auto-portant.
\paragraph{Limites.} Heuristique pour $k'$ ; index d'overflow basé sur première occurrence (peut être amélioré) ; pas d'optimalité garantie ; cas non rentables pour petits tableaux ou débordement massif ; dépendance à la distribution des valeurs.
\paragraph{Validité des mesures.} JIT/GC/thermique/OS ; répéter et lisser. Mentionner les jeux de données et les seeds.
\paragraph{Sécurité/robustesse.} Vérifier les bornes sur \verb|get(i)| ; tolérance aux fichiers corrompus (MAGIC/VERSION/TYPES), masques sûrs.

\paragraph{Difficultés rencontrées et correctifs.}
\begin{itemize}
	\item \textbf{Débordement} : correction d'un défaut lors de la décompression (extraction \emph{après} masquage du champ à \texttt{largeurChamp}) qui pouvait conduire à des erreurs (par ex. exceptions ou incohérences d'intégrité) ;
	\item \textbf{Structures} : remplacement de collections \texttt{ArrayList} par des \texttt{int[]} pour éviter surcoûts et simplifier l'interface binaire ;
	\item \textbf{Affichage} : correction de l'affichage de tableaux (éviter la forme \verb|[I@xxxx|) pour une lecture claire du contenu ;
	\item \textbf{API} : ajout de getters/setters cohérents dans les classes concernées ;
	\item \textbf{Pédagogie} : messages explicatifs quand la compression est non pertinente (notamment pour \emph{SansChevauchement}).
\end{itemize}
